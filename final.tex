%Calculus Homework
\documentclass[a4paper, 12pt]{article}

%================================================================================
%Package
    \usepackage{amsmath, amsthm, amssymb, latexsym, mathtools, physics, cancel, amsfonts}
    \usepackage{dsfont, txfonts, soul, stackrel, tikz-cd, graphicx, titlesec, etoolbox}
    \DeclareGraphicsExtensions{.pdf,.png,.jpg}
    \usepackage{fancyhdr}
    \usepackage[shortlabels]{enumitem}
    \usepackage[pdfmenubar=true, pdfborder  ={0 0 0 [3 3]}]{hyperref}
    \usepackage{kotex}

%================================================================================
\usepackage{verbatim}
\usepackage{physics}
\usepackage{makebox}
\usepackage{pst-node}

%================================================================================
%Layout
    %Page layout
    \addtolength{\hoffset}{-50pt}
    \addtolength{\headheight}{+10pt}
    \addtolength{\textwidth}{+75pt}
    \addtolength{\voffset}{-50pt}
    \addtolength{\textheight}{+75pt}
    \newcommand{\Space}{1em}
    \newcommand{\Vspace}{\vspace{\Space}}
    \newcommand{\ran}{\textrm{ran }}
    \setenumerate{listparindent=\parindent}

%================================================================================
%Statement
    \newtheoremstyle{Mytheorem}%
    {1em}{1em}%
    {\slshape}{}%
    {\bfseries}{.}%
    { }{}

    \newtheoremstyle{Mydefinition}%
    {1em}{1em}%
    {}{}%
    {\bfseries}{.}%
    { }{}

    \theoremstyle{Mydefinition}
    \newtheorem{statement}{Statement}
    \newtheorem{definition}[statement]{Definition}
    \newtheorem{definitions}[statement]{Definitions}
    \newtheorem{remark}[statement]{Remark}
    \newtheorem{remarks}[statement]{Remarks}
    \newtheorem{example}[statement]{Example}
    \newtheorem{examples}[statement]{Examples}
    \newtheorem{question}[statement]{Question}
    \newtheorem{questions}[statement]{Questions}
    \newtheorem{problem}[statement]{Problem}
    \newtheorem{exercise}{Exercise}[section]
    \newtheorem*{comment*}{Comment}
    %\newtheorem{exercise}{Exercise}[subsection]

    \theoremstyle{Mytheorem}
    \newtheorem{theorem}[statement]{Theorem}
    \newtheorem{corollary}[statement]{Corollary}
    \newtheorem{corollaries}[statement]{Corollaries}
    \newtheorem{proposition}[statement]{Proposition}
    \newtheorem{lemma}[statement]{Lemma}
    \newtheorem{claim}{Claim}
    \newtheorem{claimproof}{Proof of claim}[claim]
    \newenvironment{myproof1}[1][\proofname]{%
  \proof[\textit Proof of problem #1]%
}{\endproof}

%================================================================================
%Header & footer
    \fancypagestyle{myfency}{%Plain
    \fancyhf{}
    \fancyhead[L]{}
    \fancyhead[C]{}
    \fancyhead[R]{}
    \fancyfoot[L]{}
    \fancyfoot[C]{}
    \fancyfoot[R]{\thepage}
    \renewcommand{\headrulewidth}{0.4pt}
    \renewcommand{\footrulewidth}{0pt}}

    \fancypagestyle{myfirstpage}{%Firstpage
    \fancyhf{}
    \fancyhead[L]{}
    \fancyhead[C]{}
    \fancyhead[R]{}
    \fancyfoot[L]{}
    \fancyfoot[C]{}
    \fancyfoot[R]{\thepage}
    \renewcommand{\headrulewidth}{0pt}
    \renewcommand{\footrulewidth}{0pt}}

    \pagestyle{myfency}

%================================================================================

%***************************
%*** Additional Command ****
%***************************

\DeclareMathOperator{\cl}{cl}
\DeclareMathOperator{\co}{co}
\DeclareMathOperator{\ball}{ball}
\DeclareMathOperator{\wk}{wk}
\DeclareMathOperator{\Ric}{Ric}
\DeclareMathOperator{\Ad}{Ad}
\DeclareMathOperator{\ad}{ad}
\DeclarePairedDelimiter{\ceil}{\lceil}{\rceil}
\DeclarePairedDelimiter\floor{\lfloor}{\rfloor}
\newcommand{\intprodl}{%
    \mathbin{\scalebox{1.5}{$\lrcorner$}}%
}
\newcommand{\quotZ}[1]{\ensuremath{\mathbb{Z}/p^{#1}\mathbb{Z}}}
\newcommand*{\vertbar}{\rule[-1ex]{0.5pt}{2.5ex}}
\newcommand*{\horzbar}{\rule[.5ex]{2.5ex}{0.5pt}}
%================================================================================
%Document
\begin{document}
\thispagestyle{myfirstpage}
\begin{center}
    \Large{FINAL}
\end{center}
박성빈, 수학과\\

\noindent \textbf{1}
Note that $M$ is compact, so the integral of form is well-defined. If there exists a diffeomorphism $\phi:M\rightarrow M$ such that $\phi^*\omega_1 = \omega_0$, then
\begin{equation*}
    \int_M \omega_0 = \int_M \phi^*\omega_1 = \int_{\phi(M)} \omega_1 = \int_{M} \omega_1.
\end{equation*}
Therefore, I'll show the converse. Assume $\int_M \omega_0=\int_M \omega_1$. Since $M$ is orientable and compact, it means $[\omega_0-\omega_1]=[0]\in H_{dR}^n(M)$, so there exists $\beta$ such that $d\beta = \omega_0-\omega_1$.

Let's define $\omega_t = (1-t)\omega_0+t\omega_1$. Since $\omega_0$ and $\omega_1$ are volume form, for any point $p\in M$ with coordinate $x$ compatible with the orientation, if I write $\omega_0$ and $\omega_1$ in the coordinate form by
\begin{equation*}
\begin{split}
    \omega_0 &= fdx^1\wedge\cdots \wedge dx^n\\
    \omega_1 &= gdx^1\wedge\cdots \wedge dx^n,
\end{split}
\end{equation*}
then $f(p),g(p)>0$, and it means $(1-t)f(p)+tg(p)>0$ for all $p$. It shows that $\omega_t$ is again volume form. Also, it satisfies $\dv{t}\omega_t = -d\beta$. To utilize Moser's trick, I'll find a collection of diffeomorphisms $\phi_t$ such that
\begin{equation*}
    (\phi_t)^*\omega_t = \omega_0.
\end{equation*}
Writing $X_t = \dv{\phi_t}{t}\circ \phi_t$,
\begin{equation*}
    \dv{t}(\phi_t)^*\omega_t = (\phi_t)^*(L_{X_t}\omega_t + d\beta) = (\phi_t)^*(d(X_t\intprodl \omega_t) + d\beta) = 0.
\end{equation*}
Therefore, I need to solve the equation
\begin{equation*}
    X_t\intprodl \omega_t = -\beta.
\end{equation*}
Since $\omega_t$ is volume form, we can find $X_t\in \Gamma(TM)$ for $t\in [0,1]$ and by computing the flow of time dependent vector field $X_t$, we get $\phi_t$ for $t\in[0,1]$; since $M\times [0,1]$ is compact, we can compute the flow of $X_t$ with initial condition by the identity map at $t=0$ on $M\times[0,1]$ using the time-independent vector field theory, and extract the solution $\phi_t(x)$ on $M$ for $t\in [0,1]$. Now, $\phi_1$ is the desired diffeomorphism making $(\phi_1)^*\omega_1 = \omega_0$.\\

\noindent \textbf{2}
\begin{enumerate}
    \item[(a)]For $\xi^1,\xi^2,\eta^1,\eta^2\in \mathfrak{g}$ and $\nu\in\mathfrak{g}^*$, assume $\xi^1_{\mathfrak{g}^*}(\nu) = \xi^2_{\mathfrak{g}^*}(\nu)=v_1$ and $\eta^1_{\mathfrak{g}^*}(\nu) = \eta^2_{\mathfrak{g}^*}(\nu)=v_2$, then by identifying $T_\nu \mathfrak{g}^*\simeq \mathfrak{g}^*$
    \begin{equation*}
        \xi^1_{\mathfrak{g}^*}(\nu)(\eta) = -\nu(\ad_{\xi^1}\eta)=-\nu(\ad_{\xi^2}\eta) = \xi^2_{\mathfrak{g}^*}(\nu)(\eta)
    \end{equation*}
    for any $\eta\in\mathfrak{g}$. Therefore, $\nu([\xi^1, \eta^1])=\nu([\xi^2, \eta^1])$. By the same computation, we get $\nu([\xi^2, \eta^1]) = \nu([\xi^2, \eta^2])$, so $\langle \nu, [\xi^1, \eta^1]\rangle = \langle \nu, [\xi^2, \eta^2]\rangle$. (The proof for non-degeneracy is on Lecture note.)
    \item[(b)] For $g\in G$,
    \begin{equation*}
    \begin{split}
        (\Phi_g)^*\omega_\mu(v_1, v_2) &= \omega_\mu(d\Phi_g(v_1), d\Phi_g(v_2)) \\
        &=\omega_\mu\left(d\Phi_g\left(\xi_{\mathfrak{g}^*}\left(\Phi_{g^{-1}}\Phi_{g}(\nu)\right)\right),d\Phi_g\left(\eta_{\mathfrak{g}^*}\left(\Phi_{g^{-1}}\Phi_{g}(\nu)\right)\right)\right)\\
        &=\omega_\mu\left((\Ad_g\xi)_{\mathfrak{g}^*}\left(\Phi_{g}(\nu)\right),(\Ad_g\eta)_{\mathfrak{g}^*}\left(\Phi_{g}(\nu)\right)\right)\\
        &=\left\langle \Ad_{g^{-1}}^*(\nu), [\Ad_g\xi, \Ad_g\eta] \right\rangle\\
        &=\left\langle \Ad_{g^{-1}}^*(\nu), \Ad_g[\xi, \eta] \right\rangle\\
        &=\left\langle \nu, \Ad_{g^{-1}}\Ad_g[\xi, \eta] \right\rangle\\
        &=\left\langle \nu, [\xi, \eta] \right\rangle.
    \end{split}
    \end{equation*}
    Therefore, the form is invariant under the coadjoint action of $G$.
    \item[(c)] For $v_1,v_2,v_3\in T_\nu\mathfrak{g}^*$,
    \begin{equation*}
    \begin{split}
        d\omega(v_1,v_2,v_3) &= \xi_{g^*}[\omega(\eta_{g^*}, \delta_{g^*})](\nu) - \eta_{g^*}[\omega(\xi_{g^*}, \delta_{g^*})](\nu) + \delta_{g^*}[\omega(\xi_{g^*}, \eta_{g^*})](\nu)\\
        &\phantom{=}-\omega\left([\xi_{g^*},\eta_{g^*}], \delta_{g^*}\right)(\nu)+\omega\left([\xi_{g^*},\delta_{g^*}], \eta_{g^*}\right)(\nu)-\omega\left([\eta_{g^*},\delta_{g^*}], \xi_{g^*}\right)(\nu).
    \end{split}
    \end{equation*}
    Since
    \begin{equation*}
        \left.\dv{t}\right|_{0}\left(\Phi_{\exp(t\xi)}\right)^*(\nu) = \xi_{\mathfrak{g}^*}(\nu),
    \end{equation*}
    by the coadjoint action invariant of $\omega_\mu$, the first line vanishes. For second line,
    \begin{equation*}
        \begin{split}
            -\omega\left([\xi_{g^*},\eta_{g^*}], \delta_{g^*}\right)(\nu)&+\omega\left([\xi_{g^*},\delta_{g^*}], \eta_{g^*}\right)(\nu)-\omega\left([\eta_{g^*},\delta_{g^*}], \xi_{g^*}\right)(\nu) \\
        &= \langle \nu, [[\eta_{g^*},\xi_{g^*}], \delta_{g^*}\rangle + \langle \nu, [[\xi_{g^*},\delta_{g^*}],\eta_{g^*}\rangle + \langle \nu, [\delta_{g^*},\eta_{g^*}], \xi_{g^*}\rangle=0
        \end{split}
    \end{equation*}
    by Jacobi's identity.\\
\end{enumerate}

\noindent \textbf{3}
We already check that
\begin{equation*}
    \psi_{g,\xi} = J_\xi - \left(\Ad_{g^{-1}}^*J\circ \Phi_g^{-1}\right)_\xi
\end{equation*}
is a constant function on $M$. Therefore, it is safe to set
\begin{equation*}
    \langle\sigma_{J}(g), \xi\rangle = \psi_{g,\xi}(x_0) = J_\xi(x_0) - \left(\Ad_{g^{-1}}^*J\circ \Phi_g^{-1}(x_0)\right)_\xi=J_\xi(x_0) - \left(\Ad_{g^{-1}}^*J(x_0)\right)_\xi
\end{equation*}
for all $g$. Now, set $f\in C^\infty(M, \mathfrak{g}^*)$ by
\begin{equation*}
    f(x) = J(x_0),
\end{equation*}
which is constant function, then for boundary operator in group cohomology with coadjoint action of $\mathfrak{g}^*$,
\begin{equation*}
    -(\delta f)(g) = f-g\cdot f = J(x_0) - \Ad_{g^{-1}}^*J(x_0) = \sigma_{J}(g).
\end{equation*}
It shows that $[\sigma_J(g)] = 0\in H^1(G,\mathfrak{g}^*)$, and $(M, \Phi, \omega, J)$ is a Hamiltonian $G$-space.\\

\noindent \textbf{4}
\begin{enumerate}
    \item[(a)] Let $i:S^2(1)\rightarrow \mathbb{R}^3$ be the embedding. Using the embedding, I'll identify $S^2(1)$ as a subset of $\mathbb{R}^3$ and tangent vector $u\in T_qS^2(1)$ as $u\in T_q\mathbb{R}^3$. Identifying $T_q\mathbb{R}^3\simeq \mathbb{R}^3$, we can write
    \begin{equation*}
        \omega_{S^2(1)}(u,v) = \langle q, u\times v\rangle.
    \end{equation*}
    It is non-degenerate since $q\perp u,v$, so if $u\times v\neq 0$, then $\omega_{S^2(1)}(u,v)\neq 0$. Also, it is a two form by the definition. Finally, using the parametrization $(\theta,\phi)\in(0,\pi)\times(0,2\pi)$, which is coordinate notation used in physics, such that 
    \begin{equation*}
        \varphi:(\theta,\phi)\mapsto (\sin\theta\cos\phi, \sin\theta\sin\phi, \cos\theta)
    \end{equation*}
    with image in $S^2$ minus measure zero set, we get
    \begin{equation*}
        \begin{split}
            \pdv{\varphi}{\theta} &= (\cos\theta\cos\phi, \cos\theta\sin\phi, -\sin\theta)\\
            \pdv{\varphi}{\phi} &= (-\sin\theta\sin\phi, \sin\theta\cos\phi, 0).
        \end{split}
    \end{equation*}
    It shows that
    \begin{equation*}
        \left\langle (\sin\theta\cos\phi, \sin\theta\sin\phi, \cos\theta), \pdv{\varphi}{\theta}\times \pdv{\varphi}{\phi} \right\rangle = \sin\theta.
    \end{equation*}
    Therefore, $F^*\omega_{S^2(1)} = \sin\theta d\theta\wedge d\phi$, and by integrating it, we get
    \begin{equation*}
        \int_{S^2(1)}\omega_{S^2(1)} = \int_0^{2\pi} \int_0^\pi \sin\theta d\theta d\phi = 4\pi.
    \end{equation*}
    Finally, as a two form on $2$-dim manifold, it is closed.
    \item[(b)] Let's write the basis of $\mathit{so}(3)$ by
\begin{equation*}
    \begin{split}
        L_x &= \begin{pmatrix}
        0 & 0 & 0\\
        0 & 0 & -1\\
        0 & 1 & 0
        \end{pmatrix}\\
        L_y &= \begin{pmatrix}
        0 & 0 & 1\\
        0 & 0 & 0\\
        -1 & 0 & 0
        \end{pmatrix}\\
        L_z &= \begin{pmatrix}
        0 & -1 & 0\\
        1 & 0 & 0\\
        0 & 0 & 0
        \end{pmatrix}.
    \end{split}
\end{equation*}
In this setting, we get $[L_x,L_y]=L_z$, $[L_y, L_z]=L_x$, and $[L_z, L_x] = L_y$. Therefore, we can identify $L_x\mapsto (1,0,0)$, $L_y\mapsto (0,1,0)$, and $L_z\mapsto (0,0,1)$. For $\xi = aL_x+bL_y+cL_z$ and $q=(\alpha,\beta,\gamma)$,
    \begin{equation*}
        \xi_{S^2}(q) = \left.\dv{t}\right|_{0}\exp(t\xi)q = \begin{pmatrix}
        0 & -c & b\\
        c & 0 & -a\\
        -b & a & 0
        \end{pmatrix}
        \begin{pmatrix}
            \alpha\\ \beta\\ \gamma
        \end{pmatrix}=\begin{pmatrix}
            \gamma b-\beta c\\
            \alpha c-\gamma a\\
            -\alpha b + \beta a
        \end{pmatrix}
        = (a,b,c)\times (\alpha, \beta,\gamma)
    \end{equation*}
    \item[(c)] Let's define $J_\xi(q) = q\cdot \xi=\langle q, \xi\rangle$ for $q\in S^2$ and $\xi\in \mathit{so}(3)$ by identifying each in $\mathbb{R}^3$, i.e. for $q=(q_1,q_2,q_3)$ and $\xi=aL_x+bL_y+cL_z$,
    \begin{equation*}
        J_\xi(q) = aq_1+bq_2+cq_3.
    \end{equation*}
    Then for a curve $\gamma:(-\epsilon,\epsilon)\rightarrow S^2$ such that $\gamma(0)=q$ and $\gamma'(0) = \eta$,
    \begin{equation*}
    \begin{split}
        (dJ_\xi)(\eta) = \left.\dv{t}\right|_{t=0}J_\xi(\gamma(t)) = \left.\dv{t}\right|_{t=0}\gamma(t)\cdot \xi = \eta\cdot \xi = \langle q, (\xi\times q)\times \eta\rangle.
    \end{split}
    \end{equation*}
    I'll check the equivariance. For $A\in \mathrm{SO}(3)$, $\Phi_A(q) = Aq$, so
    \begin{equation*}
        \langle J\circ \Phi_A(q), \xi\rangle  = \langle A(q), \xi\rangle = \langle q, A^T\xi\rangle = \langle J(q), A^{-1}\xi\rangle
    \end{equation*}
    For $\xi\in \mathit{so}(3)$, 
    \begin{equation*}
        \Ad_{A}\xi = A\xi A^{-1}
    \end{equation*}
    by identifying $3\times 3$ matrix multiplication. Using the previous computation by replacing $E\mapsto L_z$, $F\mapsto L_y$, and $G\mapsto L_x$, for $\xi = aL_x+bL_y+cL_z$,
    \begin{equation*}
        \Ad_{A}\xi = A\begin{pmatrix}
            a\\b\\c
        \end{pmatrix}.
    \end{equation*}
    It shows for any $\xi$,
\begin{equation*}
    \langle J\circ \Phi_A(q), \xi\rangle  = \langle J(q), A^{-1}\xi\rangle = \langle J(q), \Ad_{A^{-1}}\xi\rangle = \langle \Ad_{A^{-1}}^*J(q), \xi\rangle.
\end{equation*}
\end{enumerate}

\noindent \textbf{5}
\begin{enumerate}
    \item[(a)] By repeating the computation for chart
    \begin{equation*}
        \varphi:(\theta,\phi)\mapsto (r\sin\theta\cos\phi, r\sin\theta\sin\phi, r\cos\theta)
    \end{equation*}
    with image in $S^2(r)$ minus measure zero set,
    \begin{equation*}
        \left\langle (r\sin\theta\cos\phi, r\sin\theta\sin\phi, r\cos\theta), \pdv{\varphi}{\theta}\times \pdv{\varphi}{\phi} \right\rangle = r^3\sin\theta.
    \end{equation*}
    Let's define $\omega_{S^2(r)}(u,v) = \frac{1}{r}\langle q, u\times v\rangle$ for $u,v\in T_qS^2(r)$, then by the same computation in problem 4, we get the area $4\pi r^2$.
    
    \item[(b)] For $q\in \prod_{i=1}^n S^2(r_i)$ and $\eta \in T_q\prod_{i=1}^n S^2(r_i)$.
    \begin{equation*}
    \begin{split}
        q&=(q_1, \ldots, q_n)\\
        \eta&=((\eta^1_1, \eta^2_1,\eta^3_1), \ldots, (\eta^1_n, \eta^2_n,\eta^3_n))
    \end{split}
    \end{equation*}
    identifying $T_q\prod_{i=1}^n S^2(r_i)\simeq \prod_{i=1}^n T_{q_i}S^2(r_i)$.
    Now, define a function
    \begin{equation*}
        J(\vec{q}) = \sum_{i=1}^n q_i,
    \end{equation*}
    explicitly, for the basis $L_x^*$, $L_y^*$, and $L_z^*$,
    \begin{equation*}
        J(\vec{q}) = \sum_{i=1}^n (q_i^1L_x^* + q_i^2L_y^* + q_i^3L_z^*).
    \end{equation*}
    Then, for $\xi=(\alpha,\beta,\gamma)\in \mathit{so}(3)$ and $\eta \in T_{q}\prod_{i=1}^n S^2(r_i)$,
    \begin{equation*}
        dJ_\xi(\eta) =\langle dJ(\eta), \xi\rangle = \sum_{i=1}^n (\alpha \eta_i^1 + \beta \eta_i^2 + \gamma \eta_i^3).
    \end{equation*}
    Let give an action of $A\in \mathrm{SO}(3)$ to $q\in\prod_{i=1}^n S^2(r_i)$ by
    \begin{equation*}
        A\cdot q = \left(A\cdot q_1, \ldots, A\cdot q_n\right).
    \end{equation*}
    Then for $\xi\in \mathit{so}(3)$,
    \begin{equation*}
    \begin{split}
        \xi_{\prod_{i=1}^n S^2(r_i)}(q) &= \left.\dv{t}\right|_{0}\exp(t\xi)\cdot q \\
        &= \left.\dv{t}\right|_{0}\left(\exp(t\xi)\cdot q_1, \ldots, \exp(t\xi)\cdot q_n\right)\\
        &= \left(\xi_{S^2(r_1)}(q_1), \ldots, \xi_{S^2(r_n)}(q_n)\right)\\
        &=\left(\xi\times q_1, \ldots, \xi\times q_n\right).
    \end{split}
    \end{equation*}
    Therefore,
    \begin{equation*}
    \begin{split}
        \omega(\xi_{\prod_{i=1}^n S^2(r_i)}(q),\eta) &=\sum_{i=1}^n \frac{1}{r_i}\omega_{S^2(r_i)}(\xi_{S^2(r_i)}(q_i),\eta_i)= \sum_{i=1}^n \frac{1}{r_i^2}\langle q_i, (\xi\times q_i)\times \eta_i\rangle\\
        &=\sum_{i=1}^n \frac{1}{r_i}\langle q_i, q_i(\xi\cdot \eta_i)\rangle=\sum_{i=1}^n (\xi\cdot \eta_i) = dJ_\xi(\eta).
    \end{split}
    \end{equation*}
    It shows that $J$ is a moment map. Also, it satisfies $\Ad^*$ equivariant: for $A\in \mathrm{SO}(3)$,
    \begin{equation*}
        \langle J(A\cdot q), \xi\rangle = \sum_{i=1}^n \langle Aq_i, \xi\rangle = \sum_{i=1}^n \langle q_i, A^{-1}\xi\rangle = \langle J(q), \Ad_{A^{-1}}\xi\rangle = \langle \Ad^*_{A^{-1}}J(q), \xi\rangle.
    \end{equation*}
    \item[(c)] I'll show that the Hamiltonian vector field on $\prod_{i=1}^n S^2(r_i)$ is
    \begin{equation*}
        X_{f_k}(q) = \left(\left(\sum_{i=1}^{k}q_i\right)\times q_1, \cdots, \left(\sum_{i=1}^{k}q_i\right)\times q_{k}, 0, \cdots, 0\right).
    \end{equation*}
    As a rotation matrix $A\in \mathrm{SO}(3)$, for any $x\in J^{-1}(0)$, $Ax\in J^{-1}(0)$ and $Ax = 0$ for all $x$ means that $A=\mathrm{Id}$ since we can rotate $x$ using $B\in \mathrm{SO}(3)$ to have a axis which is not eigenvector of $A$ before acting $A$ to $x$, which means that $\mathrm{SO}(3)$ freely acts on $J^{-1}(0)$. The rest follows from the lecture we did in the class.
    
    For any $\eta\in T_qM$,
    \begin{equation*}
        df_k(\eta) = \frac{1}{2}\left.\dv{t}\right|_{0}\abs{\sum_{i=1}^{k}(\exp_{q_i}(t\eta_i))}^2 = \sum_{i=1}^{k}\sum_{j=1}^{k}q_i\cdot \eta_j.
    \end{equation*}
    where $\exp_{q_i}$ is the exponential map generated generated by the natural Riemannian metric. Also,
    \begin{equation*}
        \omega\left(X_{f_k}(q),\eta\right) = \sum_{j=1}^{k}\frac{1}{r_j^2}\left\langle q_j, \left(\sum_{i=1}^{k}q_i\right)\times q_j)\times \eta_j\right\rangle = \sum_{j=1}^{k}\left(\left(\sum_{i=1}^{k} q_i\right)\cdot \eta_j\right).
    \end{equation*}
    By the non-degeneracy of $\omega$, the corresponding Hamiltonian vector field of $f_k$ is $X_{f_k}$ set above.
    Finally, let's compute the Poisson bracket. WLOG, I'll assume $s<t$ for $f_s,f_t$.
    \begin{equation*}
    \begin{split}
        \{f_s, f_t\}(q) &= \omega(X_{f_s}, X_{f_t})(q) = \sum_{k=1}^{s+1}\frac{1}{r_k^2}\left\langle q_k, \left(\left(\sum_{i=1}^{s+1}q_i\right)\times q_k\right)\times \left(\left(\sum_{j=1}^{t+1}q_j\right)\times q_k\right)\right\rangle\\
        &=\sum_{k=1}^{s+1}\frac{1}{r_k^2}\left\langle q_k, q_k\left(q_k\cdot\left(\left(\sum_{i=1}^{s+1}q_i\right)\times \left(\sum_{j=1}^{t+1}q_j\right)\right)\right)\right\rangle\\
        &=\left(\sum_{k=1}^{s+1}q_k\right)\cdot\left(\left(\sum_{i=1}^{s+1}q_i\right)\times \left(\sum_{j=1}^{t+1}q_j\right)\right)\\
        &=0.
    \end{split}
    \end{equation*}
    Now, let's figure out how $X_{f_k}$ looks like in $(M_r, \omega_r)$. Let $(f_k)_{r}$ be the descent of $f_k$ to $M_r$; note that $f_k$ is $G$-invariant. Since $X_{f_1}(q) = (q_1\times q_1, 0,\ldots, 0)=0$, we can safely ignore $f_1$ case. Since $J(M_r) = 0$, $\sum_{i=1}^n q_i = 0$, so $X_{f_k}|_{J^{-1}(0)} = 0$, and we can also ignore $f_n$ case. Since $f_k$ is $G$-invariant, we can fix $q_n = (0,0,r_n)$ in  $M_r$ by choosing appropriate $A\in \mathrm{SO}(3)$ making $q_n$ maps to $(0,0,r_n)$. In this setting, we get
    \begin{equation}
        [q] = [(q_1, \ldots, q_{n-1}, (0,0,r_n)],
    \end{equation}
    i.e. $\sum_{i=1}^{n-1}q_{i} = (0,0,-r_n)$. Therefore, $q_{n-1} = (0,0,-r_n) - \sum_{i=1}^{n-2}q_{i}$, and it matches with the dimension analysis: $\dim M_r = 2n-6$. Finally, $d\pi(X_{f_i})$ are non-zero in $M_r$ for submersion $\pi:J^{-1}(0)\rightarrow J^{-1}(0)/\mathrm{SO}(3)$ and linearly independent as $n-i$ positions from back is zero in $X_{f_i}$ for $2\leq i\leq n-2$. Therefore, $X_{(f_i)_r}$ are linearly independent and $\{(f_{i})_r\}_{i=2}^{n-2}$ are in involution. As a $n-3$ functions in involution in $2n-6$ dimensional symplectic manifold, it is a completely integrable system of $(M_r, \omega_r)$.
\end{enumerate}
%________________________________________________________________________
\end{document}

%================================================================================